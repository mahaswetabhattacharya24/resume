%-------------------------
% Resume in Latex
% Author : Jake Gutierrez
% Based off of: https://github.com/sb2nov/resume
% License : MIT
%------------------------

\documentclass[letterpaper,10pt]{article}

\usepackage{latexsym}
\usepackage[empty]{fullpage}
\usepackage{titlesec}
\usepackage{marvosym}
\usepackage[usenames,dvipsnames]{color}
\usepackage{verbatim}
\usepackage{enumitem}
\usepackage[hidelinks]{hyperref}
\usepackage{fancyhdr}
\usepackage[english]{babel}
\usepackage{tabularx}
\usepackage{ragged2e}
\usepackage[usenames,dvipsnames]{color}
\usepackage[
    backend=biber,
    style=ieee,
    sorting=ydnt
]{biblatex}
\addbibresource{publications.bib}
\input{glyphtounicode}
\definecolor{darkblue}{RGB}{0,0,139}

%----------FONT OPTIONS----------
% sans-serif
% \usepackage[sfdefault]{FiraSans}
% \usepackage[sfdefault]{roboto}
% \usepackage[sfdefault]{noto-sans}
% \usepackage[default]{sourcesanspro}

% serif
% \usepackage{CormorantGaramond}
\usepackage{charter}


\pagestyle{fancy}
\fancyhf{} % clear all header and footer fields
\fancyfoot{}
\renewcommand{\headrulewidth}{0pt}
\renewcommand{\footrulewidth}{0pt}

% Adjust margins
\addtolength{\oddsidemargin}{-0.5in}
\addtolength{\evensidemargin}{-0.5in}
\addtolength{\textwidth}{1in}
\addtolength{\topmargin}{-.5in}
\addtolength{\textheight}{1.0in}

\urlstyle{same}

\raggedbottom
\raggedright
\setlength{\tabcolsep}{0in}

% Sections formatting
% Sections formatting
\titleformat{\section}{
  \vspace{-4pt}\bfseries\scshape\raggedright\large
}{}{0em}{}[\color{black}\titlerule \vspace{-5pt}]

% Ensure that generate pdf is machine readable/ATS parsable
\pdfgentounicode=1

%-------------------------
% Custom commands
\newcommand{\resumeItem}[1]{
  \item\small{
    {#1 \vspace{-2pt}}
  }
}

\newcommand{\resumeSubheading}[4]{
  \vspace{-2pt}\item
    \begin{tabular*}{0.97\textwidth}[t]{l@{\extracolsep{\fill}}r}
      \textbf{#1} & #2 \\
      \textit{\small#3} & \textit{\small #4} \\
    \end{tabular*}\vspace{-7pt}
}

\newcommand{\resumeSubSubheading}[2]{
    \item
    \begin{tabular*}{0.97\textwidth}{l@{\extracolsep{\fill}}r}
      \textit{\small#1} & \textit{\small #2} \\
    \end{tabular*}\vspace{-7pt}
}

\newcommand{\resumeProjectHeading}[2]{
    \item
    \begin{tabular*}{0.97\textwidth}{l@{\extracolsep{\fill}}r}
      \small#1 & #2 \\
    \end{tabular*}\vspace{-7pt}
}

\newcommand{\resumeSubItem}[1]{\resumeItem{#1}\vspace{-4pt}}

\renewcommand\labelitemii{$\vcenter{\hbox{\tiny$\bullet$}}$}

\newcommand{\resumeSubHeadingListStart}{\begin{itemize}[leftmargin=0.15in, label={}]}
\newcommand{\resumeSubHeadingListEnd}{\end{itemize}}
\newcommand{\resumeItemListStart}{\begin{itemize}\justifying}
\newcommand{\resumeItemListEnd}{\end{itemize}\vspace{-5pt}}

%-------------------------------------------
%%%%%%  RESUME STARTS HERE  %%%%%%%%%%%%%%%%%%%%%%%%%%%%


\begin{document}

%----------HEADING----------
% \begin{tabular*}{\textwidth}{l@{\extracolsep{\fill}}r}
%   \textbf{\href{http://sourabhbajaj.com/}{\Large Sourabh Bajaj}} & Email : \href{mailto:sourabh@sourabhbajaj.com}{sourabh@sourabhbajaj.com}\\
%   \href{http://sourabhbajaj.com/}{http://www.sourabhbajaj.com} & Mobile : +1-123-456-7890 \\
% \end{tabular*}

\begin{center}
    {\Huge \textbf{\textcolor{darkblue}{Mahasweta Bhattacharya}}} \\ \vspace{1pt}
    \small 180 Kennedy Dr $|$ Malden, MA 02148 \\ \vspace{1pt}
    \small 609-906-1583 $|$ \href{mailto:b.mahasweta24@gmail.com}{\textcolor{darkblue}{\underline{b.mahasweta24@gmail.com}}} $|$ 
    \href{https://linkedin.com/in/mahasweta-bhattacharya}{\textcolor{darkblue}{\underline{linkedin.com/in/mahasweta-bhattacharya}}} $|$
    \href{https://github.com/mahaswetabhattacharya24}{\textcolor{darkblue}{\underline{github.com/mahaswetabhattacharya24}}}
\end{center}


%-----------EDUCATION-----------
\section{Education}
  \resumeSubHeadingListStart
    \resumeSubheading
      {\textcolor{darkblue}{University at Buffalo, State University of New York}}{Buffalo, NY}
      {Doctor of Philosophy in Biomedical Engineering}{Aug. 2017 -- Jan 2023}
    \resumeSubheading
      {\textcolor{darkblue}{University at Buffalo, State University of New York}}{Buffalo, NY}
      {Master of Science in Electrical Engineering}{Aug. 2015 -- May 2017}
    \resumeSubheading
      {\textcolor{darkblue}{West Bengal University of Technology}}{Kolkata, India}
      {Bachelor of Technology in Electronics and Communication Engineering}{Aug. 2010 -- May 2014}
  \resumeSubHeadingListEnd

%-----------PROGRAMMING SKILLS-----------
\section{Skills}
\begin{itemize}[leftmargin=*, itemsep=0pt]
  \item \textcolor{darkblue}{\textbf{Machine Learning:}} Supervised/unsupervised learning, representation learning, causal inference, graph neural networks, multimodal learning, VAEs, uncertainty quantification.
  \item \textcolor{darkblue}{\textbf{LLMs \& Agentic AI:}} LLM fine-tuning (LoRA/PEFT), RAG pipelines, multi-agent orchestration (Claude/GPT), structured generation, scientific summarization, grounding and consistency checking.
  \item \textcolor{darkblue}{\textbf{Applied ML:}} End-to-end ML pipelines, feature engineering, embedding similarity search, statistical modeling, Bayesian inference, high-dimensional data analysis.
  \item \textcolor{darkblue}{\textbf{AI Systems:}} Python (PyTorch, TensorFlow, JAX), CUDA/GPU computing, Docker, HPC (Slurm).
  \item \textcolor{darkblue}{\textbf{Data Engineering:}} ETL pipelines, data harmonization/QC, SQL/Snowflake, graph databases (Neo4j), S3 data lakes.
  \item \textcolor{darkblue}{\textbf{Cloud \& Databases:}} AWS (EC2, S3, Batch), SQL, Snowflake, MySQL, S3-based data lakes, scalable storage architectures.
  % \item \textbf{BioAI \& Computational Biology:} Multi-omics integration, causal target discovery, disease network modeling, pathway/DE analysis, biomarker discovery, biological embeddings.
  % \item \textbf{Neuroscience \& Biomedical Signals:} Calcium imaging analysis, spike inference, fNIRS–EEG processing, consciousness metrics, organoid MEA analysis, frequency-domain modeling.
\end{itemize}


%-----------EXPERIENCE-----------
\section{Experience}
  \resumeSubHeadingListStart

    \resumeSubheading
      {Senior Scientist}{Sept. 2023 -- Present}
      {\textcolor{darkblue}{\textbf{Sanofi}}}{Cambridge, MA}
      \resumeItemListStart
      \resumeItem{Led the design of an \textbf{agentic-AI pipeline orchestrating Claude-based agents} to autonomously aggregate and summarize multimodal biological evidence for target credentialing; delivered scalable plain-language evidence reports, reduced manual review overhead by \textbf{$>$50\%}, and operationalized \textbf{LLM-driven reasoning in a regulated scientific workflow.}}
      \resumeItem{Designed a multi-modal foundation model integrating genetics, transcriptomics, and clinical embeddings; achieved \textbf{4$\times$ improvement in causal target recall} over genetics-only baselines and established a transferable representation space for cross-disease generalization.}
      \resumeItem{Led transcriptomic pharmacodynamics modeling to compare oral vs injectable therapies for Hidradenitis Suppurativa; identified \textbf{superior immune-pathway perturbation} for the oral candidate, enabling preclinical advancement and establishing a robust MoA modeling workflow.}
      \resumeItem{Developed a harmonized meta-analysis pipeline for public HS transcriptomes, producing a \textbf{mechanistic target-ranking framework} presented at FOCIS 2025 and forming the computational backbone for patient stratification and network modeling.}
      \resumeItem{Founded a graph-based bispecific discovery platform integrating synergy metrics, biological embeddings, and \textbf{LLM-guided evidence retrieval}; generated \textbf{5 novel bispecific target-pair candidates}.}
      \resumeItem{Built a scalable disease-mapping and indication-discovery engine scoring \textbf{232 immune indications in 3 weeks} and scaling to \textbf{17,000+ phenotypes}, enabling computational repurposing and whitespace identification.}
      \resumeItem{Developed an explainable AI-driven target-discovery engine generating \textbf{90+ hypotheses} and advancing \textbf{7 novel targets} into preclinical evaluation; incorporated causal scoring, embedding similarity, and LLM-augmented evidence synthesis.}
      \resumeItem{Co-led an automated multimodal target-credentialing platform supporting \textbf{30+ therapeutic programs} and enabling \textbf{3 preclinical nominations}; introduced modules for causal inference, uncertainty quantification, and prospective validation.}
      \resumeItemListEnd
      
% -----------Multiple Positions Heading-----------
%    \resumeSubSubheading
%     {Software Engineer I}{Oct 2014 - Sep 2016}
%     \resumeItemListStart
%        \resumeItem{Apache Beam}
%          {Apache Beam is a unified model for defining both batch and streaming data-parallel processing pipelines}
%     \resumeItemListEnd
%    \resumeSubHeadingListEnd
%-------------------------------------------

    \resumeSubheading
  {Postdoctoral Fellow, Radiation Oncology}{Jan 2023 - Sept 2023}
  {\textcolor{darkblue}{\textbf{Johns Hopkins University School of Medicine}}}{Baltimore, MD}
  \resumeItemListStart
    \resumeItem{Developed a GPU-accelerated double-Gaussian proton dose engine as a Python package; achieved \textbf{0.28s} mean patient-plan computation vs \textbf{4.68s} Monte Carlo with \textbf{96\% 3D-gamma agreement (2\%/2mm)}, demonstrating physics-informed acceleration at ML-scale throughput.}
    \resumeItem{Architected an end-to-end beam-modeling and validation pipeline from \textbf{98 pristine Bragg-peak energies}, benchmarking against measurements, heterogeneous digital phantoms, and multi-site patient plans; built uncertainty profiles highlighting limits in highly heterogeneous regions.}
    \resumeItem{Advanced a \textbf{deep reinforcement learning framework} for VMAT machine-parameter optimization, producing deliverable prostate plans in \textbf{3.3s} and automated TPS refinement in \textbf{77s}; on a 15-patient external cohort, RL+TPS reduced mean rectum dose vs clinical plans (\textbf{17.4 vs 21.0 Gy}, p=0.024) while maintaining target coverage.}
    \resumeItem{Integrated RL-generated optimization with a commercial treatment-planning system to create a \textbf{hybrid automated + human-in-the-loop pipeline}, operationalizing ML methods into clinical workflows and demonstrating production-grade feasibility.}

  \resumeItemListEnd

    \resumeSubheading
  {Research Assistant}{2017 -- 2023}
  {\textcolor{darkblue}{\textbf{State University of New York at Buffalo}}}{Buffalo, NY}
  \resumeItemListStart
    \resumeItem{Analyzed longitudinal two-photon calcium imaging to reconstruct session-wise functional connectomes during motor-skill learning; identified a \textbf{biphasic rewiring trajectory} (early expansion then pruning) and discovered stable L2/3 hub assemblies encoding movement.}
    \resumeItem{Integrated portable fNIRS and EEG with electric-field–informed GLMs to model cerebellar tDCS response in stroke; uncovered \textbf{0.07--0.13 Hz hemoglobin–EEG coupling signatures} predictive of responders and interpretable non-response patterns.}
    \resumeItem{Developed the \textbf{FARCI} MATLAB toolbox for fast connectome inference using OASIS spike deconvolution and partial correlations; outperformed existing spike-inference algorithms on NCC and NAOMi benchmarks in accuracy and scalability.}
    \resumeItem{Demonstrated that low-frequency coupling between frontal HbO (0.07--0.13 Hz) and EEG (1--12 Hz) tracks consciousness in acute brain injury; AMICA embeddings + k-NN achieved \textbf{$>$90\% accuracy} distinguishing conscious vs unresponsive patients and \textbf{$>$99\% accuracy} predicting failure to recover consciousness.}
    \resumeItem{Built a multi-sensing cerebral organoid platform integrating Vis–NIR spectroscopy (mitochondrial CCO) with MEA-derived spectral exponents; quantified maturation-linked decreases in \textbf{30--50 Hz slope} and CCO activity, enabling functional screening of metabolic–electrophysiological coupling.}
    \resumeItem{Modeled photothermal vs photobiomodulation mechanisms using Monte Carlo photon transport + bioheat modeling at 630/700/810 nm; predicted \textbf{$<$0.25$^\circ$C scalp} and \textbf{$<$0.04$^\circ$C cortical} heating at 810 nm, supporting safety and highlighting CCO-mediated photobiomodulation as the primary mechanism.}

  \resumeItemListEnd
  \resumeSubHeadingListEnd


%-----------PUBLICATIONS-----------
\section{Publications}
\nocite{*}
\printbibliography[heading=none]

%-----------TALKS & PRESENTATIONS-----------
\section{Talks \& Presentations}
  \subsection*{\textcolor{darkblue}{As Presenter / Speaker}}
  \begin{itemize}[leftmargin=*, itemsep=0pt]
    \item HubXChange AI in Drug Discovery, Boston, MA \hfill 2025
    \item \textit{Development of a Python package for fast GPU-based proton pencil beam dose calculation}, American Association of Physicists in Medicine (AAPM), Houston, TX \hfill 2023
    \item \textit{Stability of Motor Cortex Decoders during Learning}, Society for Neuroscience (SfN) Annual Meeting, San Diego, CA \hfill 2022
    \item \textit{Does Learning Alter Neural Decoders of Motor Cortex?}, Society for Neuroscience Global Connectome, Virtual \hfill 2021
    \item \textit{Computational modeling of neural activation during transcranial photothermal stimulation}, Society for Neuroscience Annual Meeting, Chicago, IL \hfill 2019
    \item \textit{Development of bidirectional `mini-Brain' computer interface (mBCI) to modulate functional neural circuits}, Society for Neuroscience Annual Meeting, San Diego, CA \hfill 2017
  \end{itemize}

  \subsection*{\textcolor{darkblue}{As Co-author}}
    \begin{itemize}[leftmargin=*, itemsep=0pt]
      \item \textit{MultiTIE: Sanofi’s Multi-Specific Target Immune Engine}, International Systems for Molecular Biology (ISMB) \& European Conference on Computational Biology (ECCB), Liverpool, UK \hfill 2025
      \item \textit{HSID: An Integrative Target Discovery Framework for Hidradenitis Suppurativa}, Federation of Clinical Immunology Societies (FOCIS), Boston, MA \hfill 2025
      \item \textit{Meta-analysis of Genome-wide Association Studies of Asthma Exacerbation}, Federation of Clinical Immunology Societies (FOCIS), Boston, MA \hfill 2025
      \item \textit{Understanding Neuronal Network Dynamics during Motor Skill Learning through a Model-Free Connectome Inference Method}, Society for Neuroscience Global Connectome, Virtual \hfill 2021
    \end{itemize}

%-----------HONORS & AWARDS-----------

\section{\textcolor{darkblue}{Honors \& Awards}}
\begin{itemize}[leftmargin=*, itemsep=0pt]
  \item Member, Institute of Electrical and Electronics Engineers (IEEE) \hfill 2015--Present
  \item Member, The AI Consortium \hfill 2025--Present
  \item Winning Solution, Biomedical Knowledge Graph Hackathon, BioLabs Heidelberg, Germany \& Sanofi \hfill 2025
  \item Finalist, 3 Minute Thesis Competition \hfill 2022
\end{itemize}


%-----------TEACHING-----------
\section{Teaching}

\subsection*{\textcolor{darkblue}{Course Instructor}}
\begin{itemize}[leftmargin=*, itemsep=0pt]
  \item \textit{Biomedical Circuits and Signals} \hfill Fall 2017, 2018, 2019
  \item \textit{Biomedical Engineering Biosignals Lab} \hfill Spring 2018, 2019
\end{itemize}

%-----------SERVICE-----------
\section{Professional Service}
\subsection*{\textcolor{darkblue}{Peer Review}}
\begin{itemize}[leftmargin=*, itemsep=0pt]
  \item Applied Sciences (MDPI) \hfill 2023--Present
  \item BioMedInformatics (MDPI) \hfill 2023--Present
  \item Engineering Applications of Artificial Intelligence (Elsevier) \hfill 2023--Present
  \item Heliyon (Elsevier) \hfill 2023--Present
  \item International Journal of Molecular Sciences (MDPI) \hfill 2024--Present
  \item Mathematics (MDPI) \hfill 2024--Present
  \item Symmetry (MDPI) \hfill 2024--Present
  \item Genes (MDPI) \hfill 2023--Present
\end{itemize}

%-----------SERVICE-----------
\subsection*{\textcolor{darkblue}{Departmental \& University Service}}
\begin{itemize}[leftmargin=*, itemsep=0pt]
  \item Graduate Student Ambassador, Dept.\ of Biomedical Engineering, State University of New York at Buffalo \hfill 2018--2019
\end{itemize}


\subsection*{\textcolor{darkblue}{Community Service}}
\begin{itemize}[leftmargin=*, itemsep=0pt]
  \item MIT Undergraduate Practice Opportunities Program (MIT UPOP), Mentor Panelist \hfill 2024--Present
  \item Hopkins Biotech Network Mentor Match Program, Mentor Panelist \hfill 2024
\end{itemize}



%-------------------------------------------
\end{document}
